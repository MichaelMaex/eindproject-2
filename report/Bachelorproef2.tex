\documentclass[11pt]{article}
\usepackage[utf8]{inputenc} %Input encoding

\usepackage{amsmath} %Extra math symbols and operators
\usepackage{amssymb}
\usepackage{amsthm}
\usepackage{eufrak}

\usepackage{bm} %Bold symbols

\usepackage{graphicx} %Images

\usepackage{tikz-cd} %Diagrams

\usepackage{enumitem} %Enumerate Labels

\usepackage[margin=1in]{geometry} %Adjust Margins

\usepackage{siunitx} % easily format numbers with dimensions

%\usepackage{fancyhdr} %Name on every page
%\pagestyle{fancy}
%\lhead{Rune Buckinx}

\usepackage{mdframed}
\newmdtheoremenv{definition}{Definition}[section]
\newmdtheoremenv{question}{Question}[section]
\newmdtheoremenv{lemma}{Lemma}[section]
\newmdtheoremenv{proposition}{Proposition}[section]
\newtheorem{example}{Example}[section]
\newtheorem{remark}{Remark}[section]
\newmdtheoremenv{theorem}{Theorem}[section]
\newmdtheoremenv{exercise}{Exercise}

\usepackage{import}
\usepackage{xifthen}
\usepackage{pdfpages}
\usepackage{transparent}

\newcommand{\incfig}[1]{
    \def\svgwidth{\columnwidth}
    \import{./figures/}{#1.pdf_tex}
}

\usepackage[backend=biber]{biblatex}
\addbibresource{references.bib}
%\tikzset{component/.style={draw,thick,circle,fill=white,minimum size =0.75cm,inner sep=0pt}}

\title{Modeling of MHD waves in the solar corona}
\author{Micha\"el Maex \and Rune Buckinx}
\date{March 2020}

\begin{document}

\maketitle

\begin{abstract}
    This paper presents a summary of research for a bachelor's thesis on the modelling of magnetohydrodynamic waves in the solar corona using numerical simulations via the open source code PLUTO. 
\end{abstract}
\tableofcontents
\newpage
\section{Introduction}
The Sun's corona, or solar corona, is the outer layer of the Sun's atmosphere, which consists of fully ionized material due to it having very high temperatures compared to the surface of the sun. The plasma in the solar corona can be described using magnetohydrodynamic (MHD) equations, which combine the Navier-Stokes equations from fluid dynamics with the Maxwell equations from electromagnetism. This approach to model the plasma treats it as a single fluid, as opposed to the two-fluid model, which describes the ions and electrons seperately. \\

The differential/governing equations that make up the MHD model, are difficult to solve analytically. It is therefore favorable to use numerical computer modeling to study the equations, which is the main focus of this bachelor's project. The simulations will be performed in the open source code PLUTO, which is designed to solve systems of partial differential equations in astrophysical fluid dynamics \cite{mignone2011pluto}. 
\section{Magnetohydrodynamics and the solar corona}
Magnetohydrodynamics can be described by a system of partial differential equations. 
Just like ordinary fluid dynamics, these are derived from conservation laws. 
In this report we will only consider the non-relativistic approximation. 
The system of PDE's/conservations laws is the following:
\begin{alignat}{3}
    &\frac{\partial \rho}{\partial t} &&+ \nabla \cdot (\rho \mathbf v) &&= 0 \tag{mass}\label{masscont}\\
    &\frac{\partial \mathbf m}{\partial t} &&+  \nabla \cdot \bigg[\mathbf{mv - BB+ I}\bigg(p + \frac{\mathbf B^2}{2}\bigg)\bigg]^T &&= -\rho \nabla \Phi \tag{moment}\label{cauchymoment}\\
    &\frac{\partial \mathbf B}{\partial t} &&+ \nabla \times (cE) &&= 0 \tag{charge}\label{Faraday}\\
    &\frac{\partial(E_t + \rho \Phi)}{\partial t} &&+ \nabla \cdot \bigg[\bigg(\frac{\rho \mathbf v^2}{2} + \rho e + p + \rho \Phi\bigg)v + c \mathbf E \times \mathbf B\bigg] &&= 0 \tag{energy}\label{energy}.
\end{alignat}
Here $\rho$ is the plasma density, $\mathbf B$ is the magnetic field,  $\mathbf v$ the velocity, $\mathbf m= \rho \mathbf v$ the momentum density,$\Phi$ the potential, $p$ is the thermal pressure, $E_t$ is the total energy, which is given by \[
E_t, = \rho e \frac{m^2}{2\rho} + \frac{B^2}{2}
.\]  
The constants  $e, c$ are the elementary charge and the speed of light respectively.

If the environment is one where the thermal pressure and density are small deviations of some constant background pressure and density, $p_0, \rho_0$. Then the speed of sound is given by \[
c_s^2 = \gamma \frac{p_0}{\rho_0}
,\]
where $\gamma = \frac{f + 2}{f}, $ with $f$ the degrees of freedom of a particle. In a plasma $f$ is almost always $3$. 



\section{Pluto Code}
\subsection{Initial example}
Task 1
\section{Waves in magnetohydrodynamic fluids}
Task 2
\section{Interaction of MHD waves with large scale structures}
Task 3
\section{Optional: MHD waves in slab geometry}
Task 4
\end{document}
